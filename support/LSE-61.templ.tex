\documentclass[SE,toc,lsstdraft]{lsstdoc}

## This is a template file that can be used to generate LaTeX documents
## from within MagicDraw. It uses the standard lsstdoc LaTeX classes

## When importing updates in the VTL from the Word template unicode quotes
## (single and double) must be converted to simple quotes.

## After the document has been generated, it will be neccessary to run
## the fixup.py script to ensure that the embedded HTML and list directives
## are handled properly in the LaTeX.

% We use commands to make it easy to find where parameter names and units
% are defined in the tables, and to allow hyphenation.
\newcommand{\paramname}[1]{\hspace{0pt}#1}
\newcommand{\unitname}[1]{\hspace{0pt}#1}

#set($scopedPackage = $packageScope.get(0))

\setcounter{secnumdepth}{5}

%% Define the document title, authors, date and change record
\input metadata.tex

% Environment for displaying the parameter tables in
% a consistent manner. No arguments as there are no
% captions or labels.
\newenvironment{parameters}[0]{%
\setlength\LTleft{0pt}
\setlength\LTright{\fill}
\begin{small}
\begin{longtable}[]{|p{0.5\textwidth}|l|p{0.6in}|p{1.74in}@{}|}

\hline \textbf{Description} & \textbf{Value} & \textbf{Unit} & \textbf{Name} \\ \hline
\endhead

\hline \multicolumn{4}{r}{\emph{Continued on next page}} \\
\endfoot

\hline\hline
\endlastfoot
}{%
\hline
\end{longtable}
\end{small}
}



\begin{document}
\maketitle

$scopedPackage.documentation
#foreach($child in $sorter.sort($scopedPackage.getOwnedElement()))
#if($report.containsStereotype($child,"Requirement") || $report.containsStereotype($child,"interfaceRequirement") || $child.getHumanType() == 'Package')
#recursiveReqs($child,1)
#end
#end
#macro( recursiveReqs $element $depth)
#if($element.getHumanType() == 'Package')
#parseHeader($element.name,$depth)
$element.documentation
#elseif(($report.containsStereotype($element,"Requirement") || $report.containsStereotype($element,"interfaceRequirement")) && !$report.containsStereotype($element,"VerificationElement"))
#parseHeader($element.name,$depth)
#if(!$report.getStereotypeProperty($element,"Requirement","Id").isEmpty())

\label{$report.getStereotypeProperty($element,"Requirement","Id").get(0)}
\textbf{ID:} $report.getStereotypeProperty($element,"Requirement","Id").get(0)

#end
#if(!$report.getStereotypeProperty($element,"Requirement","Text").isEmpty())
#writeReqText($report.getStereotypeProperty($element,"Requirement","Text").get(0), "Specification:")
#end

#writeReqText($element.documentation, "Discussion:")
#end

#set($refines = $array.createArray())
#foreach($subRelation in $element.get_directedRelationshipOfTarget())
#if($report.containsStereotype($subRelation,"Refine"))
#set($filler = $refines.add($subRelation))
#end
#end
#if(!$refines.isEmpty())

## Note that forrow seems to be a MagicDraw extension that is designed
## specifically for Word tables. When syncing these VTL macros from the
## Word reference remember to replace forrow with foreach.

\begin{parameters}
#foreach($refine in $refines)
$refine.getSource().get(0).documentation
&
#set($length = $refine.getSource().get(0).name.length() + 3)
$refine.getSource().get(0).get_constraintOfConstrainedElement().get(0).getSpecification().getBody().get(0).substring($length)
&
\unitname{%
$refine.getSource().get(0).unit.name
}
&
\paramname{%
$refine.getSource().get(0).name
} \\\hline
#end
\end{parameters}

#end

## Get requirements from which this requirement is derived.

#set($derives = $array.createArray())
#foreach($subRelation in $element.get_directedRelationshipOfSource())
#if($report.containsStereotype($subRelation,"DeriveReqt"))
#set($filler = $derives.add($subRelation))
#end
#end
#if(!$derives.isEmpty())

\emph{Derived from Requirements:}

#foreach($derive in $derives)
#set($dername = $derive.getTarget().get(0).name)
#if(!$dername.isEmpty())
$report.getStereotypeProperty($derive.getTarget().get(0),"Requirement","Id").get(0):
$derive.getTarget().get(0).name \newline
#end
#end

#end

#foreach($subElement in $element.getOwnedElement())
#if($report.containsStereotype($subElement,"Requirement") || $report.containsStereotype($subElement,"interfaceRequirement") || $subElement.getHumanType() == 'Package')
#set($depth = $depth + 1)
#recursiveReqs($subElement,$depth)
#set($depth = $depth - 1)
#end
#end
#end
#macro( properHeading $string $depth)
#if($depth == 1)
\section{$string}
#elseif($depth == 2)
\subsection{$string}
#elseif($depth == 3)
\subsubsection{$string}
#elseif($depth == 4)
\paragraph{$string}\hfill  % Force subsequent text onto new line
#elseif($depth == 5)
\subparagraph{$string}\hfill  % Force subsequent text onto new line
#elseif($depth == 6)
\textsc{$string}

#elseif($depth == 7)
\textbf{\textit{$string}}

#elseif($depth == 8)
\textit{$string}

#else
$string

#end
#end
## This could all be done in the python fixup script rather than in VTL
#macro( parseHeader $header $depth)
#if($header.charAt(0)=='0' || $header.charAt(0)=='1' || $header.charAt(0)=='2' || $header.charAt(0)=='3' || $header.charAt(0)=='4' || $header.charAt(0)=='5' || $header.charAt(0)=='6' || $header.charAt(0)=='7' || $header.charAt(0)=='8' || $header.charAt(0)=='9')
#properHeading($header.substring($header.indexOf(" ")),$depth)
#else
#properHeading($header,$depth)
#end
#end

## See if the supplied substring exists within the supplied text. If it does
## not then write it out ourselves. Then write the element text.
#macro( writeReqText $text $substring)
#if(!$text.isEmpty())
#set ($index = $text.indexOf($substring))
#if($index < 0)
\textbf{$substring}
#end
$text
#end
#end

\bibliography{lsst}

\end{document}
